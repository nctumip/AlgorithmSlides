\documentclass{beamer}
%\usepackage{amsmath,amssymb}
%\usepackage{epsfig}
\begin{document}
\begin{frame}{}
\centerline{Binomial Heap}
\begin{itemize}
\item A kind of {\it \bf Mergeable Heaps},
\item Support operations
\begin{itemize}
\item {\sc Make-Heap()}, creates and returns a new heap containing no elements.
\item {\sc Insert($H,x$)}, insert node $x$ into heap $H$. 
\item {\sc Minimum($H$)}, returns a pointer to the minimum in $H$.
\item {\sc Extract-Min($H$)} deletes the minimum from $H$, retunrs a pointer pointing to the node.
\item {\sc Union($H_1,H_2$)}, creates and returns a new heap that contains all the nodes of heaps $H_1$ and $H_2$. 
 $H_1$ and $H_2$ are destroyed. 
\end{itemize}
\end{itemize}
\end{frame}

\begin{frame}{}
\begin{itemize}
\item Also support
\begin{itemize}
\item {\sc Decrease-Key}($H,x,k$), assign to node $x$
 within heap $H$ the new key value $k$, which is assumed to be no gerater than its current key value. 
\item {\sc Delete($H,x$)}, delete $x$ from heap $H$. 
\end{itemize}
\item {\sc Search} cannot be done fast
\item {\sc Decrease-Key} and {\sc Delete}, we assume that there is a pointer pointing to the node. 
\end{itemize}
\end{frame}

\begin{frame}{}
\begin{table} 
\begin{tabular}{lccc}
			&Binary heap	&Binomial heap & Fibonacci heap \\
Procedure        	&   (Worst-case) 	&   (worst-case) 	&  (amortized)\\
\hline
 {\sc Make-Heap }	& $\Theta(1)$                 	& $\Theta(1)$                   	& $\Theta(1)$ \\
{\sc Insert}     	& $\Theta(\lg n)$ 			& $O(\lg n)$ 				& $\Theta(1)$ \\
{\sc Minimum} 	& $\Theta(1)$ 				& $O(\lg n)$ 		 		& $\Theta(1)$  \\
{\sc Extract-Min} & $\Theta(\lg n)$				& $\Theta(\lg n)$ 	 		& $\Theta(\lg n)$ \\
{\sc Union} 	& $\Theta(n)$ 				& $O(\lg n)$ 	 			& $\Theta(1)$  \\
{\sc Decrease-Key} & $\Theta(\lg n)$ 			& $\Theta(\lg n)$  			& $\Theta(1)$  \\
{\sc Delete} 	& $\Theta(\lg n)$ 			& $\Theta(\lg n)$ 			& $O(\lg n)$  \\
\end{tabular}
\end{table}
\end{frame}

\begin{frame}{}
\centerline{\large Binomial Tree and Binomial Heap}
\begin{itemize}
\item {Binomial Tree, $B_k$}
\item An ordered tree, defined recursively
\item $B_0$ consists of a single node
\item $B_k$ consists of two binomial trees $B_{k-1}$ that are linked together.
\item Root of one is the left most child of the root of the other. 
\end{itemize}
\end{frame}

\begin{frame}{}
\centerline{\large \it Lemma, Properties of binomial trees}
\vspace{0.5cm}
For the binomial tree $B_k$,
\begin{enumerate}
\item there are $2^k$ nodes,
\item the height of the tree is $k$,
\item there are exactly $\left (  \begin{array}{c} k \\ i  \end{array}\right )$ (binomial coefficient) nodes at depth $i$
 for $i=0, 1, \ldots, k$, and 
\item the root has degree $k$, which is greater than that of any other node; moreover if the
children of the root are numbered from left to right by $k-1$, $k-1$, \ldots, 0, child $i$
is the root of the subtree $B_i$.
\end{enumerate}
\end{frame}

\begin{frame}{}
\centerline{\large Proof of the 4 Properties}
\vspace{0.5cm}
\begin{itemize}
\item By induction on $k$. 
\item The basis is the binomial tree $B_0$. 
\item Verifying that each property hoids for $B_0$ is trival. 
\item For the inductive step, we assume that the lemma holds for $B_{k-1}$
\end{itemize}
Property 1: Binomial tree $B_k$ consists of two copies of $B_{k-1}$ (each has $2^{k-1}$ nodes, from the inductive basis).
$B_k$ has $2^{k-1}$ + $2^{k-1}$ = $2^k$ nodes. 
\end{frame}

\begin{frame}{}
\centerline{\large Proof of the 4 Properties}
\vspace{0.5cm}
Property 2: $B_k$ is obtained by linking two $B_{k-1}$s together, one is the left most child of the other. 
Thus the depth of $B_k$ is one greater that $B_{k-1}$. $B_{k-1}$ has depth $(k-1)$ from the inductive basis.
Depth of $B_k$ = $(k-1) +1 =k$.
\end{frame}

\begin{frame}{}
\centerline{\large Proof of the 4 Properties}
Property 3: Let $D(k,i)$ be the number of nodes at depth $i$ of $B_k$. Depth $i$ in $B_k$
 consists of nodes in depth $i$ in one $B_{k-1}$ and depth $i-1$ in another $B_{k-1}$. They have
 $D(k-1,i)$ and $D(k-1,i-1)$ nodes respectively (induction hypothesis). 
\begin{eqnarray}
D(k,i) &=& D(k-1,i) + D(k-1,i-1) \nonumber\\
 &=& \left (  \begin{array}{c} k-1 \\ i  \end{array}\right ) + \left (  \begin{array}{c} k-1 \\ i-1  \end{array}\right ) \nonumber\\
 &=& \left (  \begin{array}{c} k \\ i  \end{array}\right ) \nonumber
\end{eqnarray}
\end{frame}

\begin{frame}{}
\centerline{\large Proof of the 4 Properties}
Property 4: Root of $B_k$ (connecting two $B_{k-1}$) has one more degree than root of $B_{k-1}$ (degree k-1). \\
Children of $B_{k-1}$ from left to right are $B_{k-2}$, $B_{k-3}$, \ldots, $B_1$, $B_0$.  
$B_k$ is obtained by connecting a $B_{k-1}$ to that $B_{k-1}$, we have the subtree from left to right
 $B_{k-1}$, $B_{k-2}$, $B_{k-3}$, \ldots, $B_1$, $B_0$. 
\end {frame}

\begin{frame}{}
{ Corrolary } The maximum degree of any node in an $n$-node binomial tree is $\lg n$.\\
{ Proof } Immediate from Properties 1 and 4 of previous lemma. 
\end{frame}

\begin{frame}{}
\centerline{\large Binomial Heap}
\vspace{0.5cm}
A binomial heap $H$ is a set of binomial trees that satisfies the {\it binomial-heap} properties. 
\begin{enumerate}
\item Each binomial tree in $H$ obeys the {\it min-heap property}: the key of a node is greater than
 or equal to the key of its parent. We say that each such tree is {\it min-heap-ordered}. 
\item For any nonnegative integer $k$, there is at most one binomial tree in $H$ whose root has degree $k$. 
\end{enumerate}
From 1, the root of a min-heap-ordered tree contains the smallest key. From 2, an $n$-node binomial heap $H$
consists of at most $\lfloor \lg n\rfloor +1$ binomial trees. 
\end{frame}

\begin{frame}{}
\centerline{\large Representing Binomial Heaps} 
\vspace {0.5cm}
\begin{itemize}
\item A binomial tree is stored in the left-child, right-sibling representation.
\item Each node has a key field and any other satellite information.
\item a node $x$ has a pointers $p[x]$ to its parent, $child[x]$ to its leftmost child, abd $sibling[x]$
 to the sibling of $x$ immediately to its right. 
\item $x$ also has a field $degree[x]$, which is the number of children of $x$. 
\end{itemize}
\end{frame}

\begin{frame}{}
\centerline{\large Representing Binomial Heaps} 
\vspace {0.5cm}
\begin{itemize}
\item Roots of the binomial trees within the binomial heaps are organized in a linked list, which we refer to as the {\it root list}. 
The degree of the roots strictly increase as we traverse the root list. 
\item a pointer $head[H]$ points to the first root in the root lost. 
\item By 2nd binomial heap property, degrees of roots are a subset of $\{ 0,1,\ldots, \lfloor \lg n\rfloor\}$ (length of the root list
is $O(\log n)$).
\end{itemize}
\end{frame}

\begin{frame}{}
\centerline{\large Operations in Binomial Heaps}
\vspace{0.5cm}
\begin{itemize}
\item Creating a new binomial heap, return $head[H]$={ Nil}, running time is $\Theta(1)$.
\item Finding the minimum key, check all roots along the root list. There are at most $\lfloor \lg n\rfloor +1 $ roots,
can be done in $O(\lg n)$ time. 
\item Uniting two binomial heaps, biasic idea is to ``merge'' tow root list, then fix the root list
to meet the 4th property of binomial tree. If there are trees have same number of nodes (two $B_{k-1}$), link them
to form a $B_k$. After that there are at most 3 $B_k$s, can be done in $O(\lg n)$ time. 
\end{itemize}
\end{frame}

\begin{frame}{}
\centerline{\large Operations in Binomial Heaps}
\vspace{0.2cm}
\begin{itemize}
\item Inserting a node, create a one node binominal heap $H'$ in $O(1)$ time, then uniting with the $n$-node binomial heap
 $H$ in $O(\lg n)$ time. 
\item Extracting the node with minimum key, 
\begin{itemize}
\item Remove the tree, let it be $B_k$, has the smallest from the root list (findi it in $O(\lg n)$ time, remove it in $O(1)$ time). 
\item Delete the root, we then have a forest, from let to right are trees $B_{k-1}$, $B_{k-2}$, \ldots, $B_0$ (properyt 4). 
\item Reverse them and link them into a root list to form a binomial heap ($O(\lg n)$) time.
\item Uniting the two binominal heap in $O(\lg n)$ time. 
\end{itemize}
\end{itemize}
\end{frame}

\begin{frame}{}
\centerline{\large Operations in Binomial Heaps}
\vspace{0.5cm}
\begin{itemize}
\item {\sc Search} is hard, so we assuome that there is a pointer pointing to node to be decreasing key or deleting. 
\item Decreasing a key, same as the binary heap, bubbling up the node toward the root of the tree, height is $O(\lg n)$. 
\item Deleting a key, make the key value to be $-\infty$ then decreasing key. The node goes to the root, then extracting
 the node with minimum key. 
\end{itemize}
\end{frame}

\begin{frame}{Fibonacci Heap}

A kind of Mergeable Heaps, support operations
\begin{itemize}
\item {\sc Make-Heap()}
\item {\sc Insert($H,x$)}
\item {\sc Minimum($H$)}
\item {\sc Extract-Min($H$)}
\item {\sc Union($H_1,H_2$}
\item {\sc Decrease-Key($H,x,k$)}
\item {\sc Delete($H,x$)}
\end{itemize}
\end{frame}

\begin{frame}{}
\begin{table} 
\begin{tabular}{lccc}
			&Binary heap	&Binomial heap & Fibonacci heap \\
Procedure        	&   (Worst-case) 	&   (worst-case) 	&  (amortized)\\
\hline
 {\sc Make-Heap }	& $\Theta(1)$                 	& $\Theta(1)$                   	& $\Theta(1)$ \\
{\sc Insert}     	& $\Theta(\lg n)$ 			& $O(\lg n)$ 				& $\Theta(1)$ \\
{\sc Minimum} 	& $\Theta(1)$ 				& $O(\lg n)$ 		 		& $\Theta(1)$  \\
{\sc Extract-Min} & $\Theta(\lg n)$				& $\Theta(\lg n)$ 	 		& $\Theta(\lg n)$ \\
{\sc Union} 	& $\Theta(n)$ 				& $O(\lg n)$ 	 			& $\Theta(1)$  \\
{\sc Decrease-Key} & $\Theta(\lg n)$ 			& $\Theta(\lg n)$  			& $\Theta(1)$  \\
{\sc Delete} 	& $\Theta(\lg n)$ 			& $\Theta(\lg n)$ 			& $O(\lg n)$  \\
\end{tabular}
\end{table}
\end{frame}

\begin{frame}{Fibonacci Heap}
\begin{itemize}
\item Good for the cases where {\sc Extract-Min} and {\sc Delete} are few, better if there are many {\sc Decrease-Key}s. 
\item {\sc Decrease-Key} can be done in $\Theta(1)$ amortized time. Case happens when we are looking for single-source shortest path in a graph. 
\item Theoretical interest. 
\end{itemize}
\end{frame}

\begin{frame}{Fibonacci Heap}
A Collection of rooted min-heap-ordered trees, not constrained to be binomial trees. Each Node $x$,
\begin{itemize}
\item $x.p$, $p$ a pointer to its parent. 
\item $x.child$, $child$ is a pointer to one of its children. 
\item $children$ of $x$ are in a circular doubly linked list, called {\it child list} of $x$. 
\item $x.degree$, {\it degree} number of nodes in the {\it child list}. 
\item {\it x.mark} indicates that whether $x$ has lost a child since the last time {\it x} was made the child of another node. 
\begin{itemize}
\item {\it x} is unmarked if it is newly created. 
\item {\it x} is unmarked if is made child of another node. 
\end{itemize}
\end{itemize}
\end{frame}

\begin{frame}{}
\begin{itemize}
\item A Fibonacci heap {\it H},  {\it X.min} points to the root of the tree that has the smallest key. This node is called
 the {\it minimum node}. 
\item The roots of all the trees are linked together into a circular, doubly linked list, called the {\it root list}. 
\item $H.n$, the number of nodes currently in the heap. 
\end{itemize}
\end{frame}

\begin{frame}{Potential Function}
\begin{itemize}
\item Given a Fibonacci heap $H$
\begin{itemize}
\item $t(H)$, the number of trees in the root list of $H$.
\item $m(H)$, the number of marked nodes in $H$. 
\item Potential function \begin{equation}\Phi(H)=t(H)+2m(H). \label{1}\end{equation}
\item The potential of a set of Fibonacci heaps is the sum of the potentials of its constituent Fibonacci heap. 
\item Assume that a unit of potential can pay for a constant amount of work. 
\end{itemize}
\item Potential of the heap in Figure 19.2 is 5+2*3=11. 
\end{itemize}
\end{frame} 

\begin{frame}{Potential Function}
\vspace{0.5cm}
\begin{itemize}
\item Assume that we start with an empty heap.
\item Above equation is non-negative at all subsequence time, the amortized cost covers the actual cost. 
\item Assume that there is known upper bound for $D(n)$ on the maximum degree of any node in an $n$-node
Fibonacci heap (for amortixed analysis purpose). 
\item  $D(n)\le \lfloor \lg n\rfloor$ at this time, state without proof. 
\end{itemize}
\end{frame}


\begin{frame}{Mergeable-Heap Operations}
%\centerline{\large }
\begin{itemize}
\item Delay work as long as possible. 
\item {\sc Insert} and {\sc Union} work on the root list, make sure $H.min$ points to the minimum. 
\item Work delay to {\sc Delete} or {\sc Extract-Min}, after deleting, {\sc Consolidate} is required. 
\item After {\sc Consolidate}, each node in the root list has a degree that is unique within the root list, thus the size of the root list is at most $D(n)+1$.
\end{itemize}
\end{frame}

%\begin{frame}{}
%\centerline{\large Mergeable-Heap Operations}
%\vspace{0.5cm}
%\begin{itemize}
%\item If an $n$-node Fibonacci heap is a collection of unordered binomial trees, $D(n)=\lg n$. 
%\item Results of {\sc Decrease-Key} and {\sc Delete} are not unordered binomial trees, 
%thus $D(n)$ shall be bounded by non-trival arguments. 
%\end{itemize}
%\end{frame}

\begin{frame}{{\sc Create} and {\sc Minimum}}
%\centerline{\large {}}
\vspace{0.5cm}
\begin{itemize}
\item Create a new Fibonacci heap, actual cost $O(1)$, potential $\Phi(H)=0$, amortized cost is $O(1)$. 
\item Finding the minimum, follow the pointer $min[H]$. Potential does not change, actual cost is $O(1)$,
thus amortized cost is $O(1)$. 
\end{itemize}
\end{frame}

\begin{frame}{Inserting a node}
%\centerline{\large }
\begin{itemize}
\item Allocate a new node $x$, $x$ is unmarked, and insert $x$ into the root list.
 Update {\it H.min} if necessary. 
\item Make no attempt to consolidate the trees within the Fibonacci heap. 
\item The amortized cost: 
\begin{itemize}
 \item $H$ the input, $H'$ the resulting Fibonacci heap
 \item $t(H')=t(H)+1$, $m(H')=m(H)$
 \item Increase in potential is $((t(H)+1+2m(H))-(t(H)+2m(H)))=1$.
 \item Actual cost is 1, thus amortized cost is $O(1)$. 
\end{itemize}
\end{itemize}
\end{frame}

\begin{frame}{Uniting two Fibonacci heaps}
%\centerline{\large }
\begin{itemize}
\item Given $H_1$, $H_2$, concatenate the root list of $H_1$ and $H_2$,
 then determine the new minimum node.
\item No consolidation of trees occurs.
\item Potential change $\Phi(H)-(\Phi(H_1)+\Phi(H_2)) =0$, since $t(H)=t(H_1)+t(H_2)$
 and $m(H)=m(H_1)+m(H_2)$. 
\item The actual cost is $O(1)$ since doubly linked list is used. 
\item Amortized cost is $O(1)$. 
\end{itemize}
\end{frame}

\begin{frame}{Extracting the minimum node}
%\centerline{\large }
\vspace{0.5cm}
\begin{itemize}
\item Delete the minimum $z$, making all $z$'s children roots of $H$. 
\item $H.min$ points to a node ($z.right$) in the root list, not necessary to be the minimum. 
\item Then reduce the number of trees in the root list, i.e., Consolidating the root list. 
\item Finally, $H.min$ points to the minimum and return $z$. 
\end{itemize}
\end{frame}

\begin{frame}{Extracting the minimum node}
%\centerline{\large }
\vspace{0.5cm}
\begin{itemize}
\item {\sc Consolidate}
\begin{enumerate}
\item Find two roots $x$ and $y$ in the root list with the same degree, where $x.key\le y.key$. 
\item Link $y$ to $x$: remove $y$ from the root list and make $y$ a child of $x$. 
\item $x.degree$ is incremented and mark on $y$ clear. 
\end{enumerate}
\item {\sc Consolidate} needs an auxiliary array $A[0..D(n(H))]$,
\item $A[i]=y$, $y$ is currently a root with $y.degree=i$. 
\end{itemize}
\end{frame}

\begin{frame}{Extracting the minimum node}
%\centerline{\large }
\begin{itemize}
\item $H$ the Fibonacci heap just prior to deleting the minimum.
\item The actual cost
\begin{itemize}
\item $O(D(n))$ contribution comes from there being at most $D(n)$ children of $z$, and last step to 
 consolidate.
\item Scan the root list and trees merging:
\begin{itemize}
\item Size of the root list before conolidation: at most $D(n)+t(H)-1$; used to be $t(H)$, ``-1''
 remove $z$, $z$ has at most $D(n)$ children that become roots of trees.
\item Each merging removes a tree, total number of mergings $<D(n)+t(H)$.
\end{itemize}
\end{itemize}
\item Actual cost is $O(D(n)+t(H))$. 
\end{itemize}
\end{frame}

\begin{frame}{Extracting the minimum node}
%\centerline{\large }
\begin{itemize}
\item Potential changes
\begin{itemize}
\item Before extracting the minimum node, $t(H)+2m(H)$.
\item After extracting the minimum node, at most $(D(n)+1)+2m(H)$ since at most $D(n)+1$ roots remains
and no nodes become marked. 
\end{itemize}
\item Amortized cost
\begin{eqnarray}
 & & O(D(n)+t(H)) + ((D(n)+1)+2m(H)-(t(H)+2m(H))) \nonumber \\
 &=& O(D(n)) + O(t(H)) -t(H) \nonumber \\
 &=& O(D(n)) \nonumber
\end{eqnarray}
\end{itemize}
\end{frame}

\begin{frame}{Decreasing a key}

%\centerline{\large }
\vspace{0.5cm}
\begin{itemize}
\item Check whether decreasing key change anything (if $x.key$ still greater than $x.p.key$, or $x$ is the root). 
\item Decreasing the key in node $x$
\item Cutting $x$, cut the link between $x$ and its parent $y$,
\item Making $x$ a root, unmark $x$,
\item Mark $y$ if $y$ is unmarked, otherwise {\sc Cascading-Cut}, i.e., cut $y$ and recursively do so. 
\item Finally, change $H.min$ if necessary (since only key of $x$ was decreased, $H.min$ remains or points to $x$). 
\end{itemize}
\end{frame}

\begin{frame}{Decreasing a key}

%\centerline{\large }
%\vspace{0.5cm}
\begin{itemize}
\item The {\it mark} field records history information of the node $x$. Suppose that he following events have happened to node $x$,
\begin{enumerate}
\item at some time, $x$ was a root,
\item then $x$ was linked to another node,
\item  two children of $x$ were removed by cuts. 
\end{enumerate}
\item $x$ becomes a new root once its second child is removed. 
\end{itemize}
\end{frame}

\begin{frame}{}
\centerline{\large Decreasing a key}
\begin{itemize}
\item Actual cost
\begin{itemize}
\item Decrease key takes $O(1)$ time +
\item Cascading cuts, if {\sc Cascading-Cut} is recursively called $c$ times, it takes $O(c)$ time
\end{itemize}
\item Changes in Potential
\begin{itemize}
\item Each recursive call of {\sc Cascading-Cut}, except the last one, cuts a marked node and clears the marked field. 
\item Afterward, there are $t(H)+c$ trees (used to be $t(H)$ trees, $c-1$ trees produced by cascading cuts,
 and $x$). 
\item At most $m(H)-c+2$ marked nodes ($c-1$ were unmarked by cascading cuts, last call of {\sc Cascading-Cut}
makrs a node)
\end{itemize}
\end{itemize}
\end{frame}

\begin{frame}{}
\centerline{\large Decreasing a key}
\vspace{0.5cm}
\begin{itemize}
\item Changes of Potential: $((t(H)+c)+2(m(H)-c+2))$ - (t(H)+2m(H)) = $4-c$.
\item Amortized cost, $O(c)+4$ - $c$ = $O(1)$. 
\end{itemize}
\end{frame}

\begin{frame}{}
{\centerline{\large Deleting a node $x$}}
\begin{itemize}
\item Decreasing the key in $x$ to $\infty$,
\item then extracting minimum.
\item The armotized cost are respectively $O(1)$ and $O(D(n))$.
\end{itemize}
\end{frame}

\begin{frame}{}
\begin{itemize}
\item Prove the amortized time of a {\sc Fib-Heap-Extract-Min} and {\sc Fib-Heap-Delete} is $O(\lg n)$,
\item To show the upper bound $D(n)$ on the degree of any node of a $n$ node Fibonacci Heap is $O(\lg n)$. 
\item If trees in Fibonacci are unordered binomial tree, we have $D(n)=\lfloor \lg n\rfloor$.
\item But {\sc Cut} make trees that are not unordered binominal tree. 
\item To prove the $\lg n$ upper bound becomes not trival. 
\end{itemize}
\end{frame}

\begin{frame}{}
\centerline{\large To bound $D(n)$}
\begin{itemize}
\item Show that because we cut a node from it parent as soon as it loses two children, $D(n)$ is $O(\lg n)$. 
\item In particular $D(n)\le \lfloor \log_{\phi}n\rfloor $, where $\phi=(1+\sqrt{5})/2$
\item Key idea of th proof, for each node $x$, define $size(x)$ to be the number of nodes, including $x$,
 in the subtree rooted at $x$. 
\item And show that $size(x)$ is exponential in $degree[x]$.
\end{itemize}
\end{frame}

\begin{frame}{}
{\bf\it Lemma} Let $x$ be in a Fibonacci heap, and suppose that $degree[x]=k$. Let
$y_1$, $y_2$, \ldots, $y_k$ denote the children on $x$ in the order in which there were linked to $x$.
Then, $degree[y_1]\ge 0$ and $degree[y_i]\ge i-2$ for $i=2, 3, \ldots, k$. \\
{\bf\it Proof} Obviously, $degree[y_1]\ge 0$. For $i\ge 2$, note that when $y_i$ was linked to $x$, all of
$y_1$, $y_2$, \ldots, $y_{i-1}$ were children of $x$, so $degree[x]=i-1$. Node $y_i$ is linked to $x$
only if $degree[x]=degree[y_i]$, so we must have had $degree[y_i]=i-1$. Since then $y_i$ could lost
one child, since it would have been cut from $x$ if it had lost two children. We conclude that $degree[y_1]\ge i-2$. 
\end{frame}

\begin{frame}{}
For $k=0,1,2,\ldots, $ the $k$th Fibonacci number is defined recursive
$$F_k=\left\{
\begin{array}{lll}
0 & {\rm \ if\ } & k=0, \\
1 & {\rm \ if\ } & k=1, \\
F_{k-1}+F_{k-2} & {\rm \ if\ } & k\ge 2. 
\end{array}
 \right.$$
The lemma gives another way to express $F_k$. \\
{\it Lemma } For all integer $k\ge 0$, $$F_{k+2}=1+\sum^k_{i=0}F_i. $$
\end{frame}

\begin{frame}{}
{\it Proof} Induction on $k$: When $k=0$
\begin{eqnarray}
1+\sum^0_{i=0}F_i &=& 1+F_0    = 1+0   =  1 \nonumber \\
   &=& F_2. \nonumber
\end{eqnarray} 
Assume the inductive hypothesis that $F_{k+1}=1+\sum^{k-1}_{i=0}F_i$, then we have
\begin{eqnarray}
F_{k+2} &=& F_k+F_{k+1} \nonumber \\
  &=& F_k + (1+\sum_{i=0}^{k-1}F_i) \nonumber \\
  &=& 1 + \sum^{k}_{i=0}F_i \nonumber
\end{eqnarray}
\end{frame}

\begin{frame}{}
Use the inequality $$F_{k+2}\ge \phi^k$$ to show the following lemma, \\
where $\phi=(1+\sqrt{5})/2=1.61803$ is the golden ratio. \\
And we shall argue that $Size[x] > F_{k+2}$.  
\end{frame}

\begin{frame}{}
{\it Lemma} Let $x$ be any node in a Fibonacci heap, and let $k=degree[x]$. Then $size(x) \ge F_{k+2}\ge \phi^k$. \\
{\it Proof} Let $s_k$ denote the minimum possible value of $size(z)$ over all nodes $z$, s.t. $degree[z]=k$. \\
$s_0=1$, $s_1=2$, and $s_2=3$. \\
Let $y_1$, $y_2$, \ldots, $y_k$ denote the children of $x$ in the order in which there were linked to $x$. We would like to compute the 
lower bound on $size(x)$. 
\begin{enumerate}
\item count 1 for $x$,
\item count 1 for the first child $y_1$ (since $size(y_1)\ge 1$). 
\end{enumerate}
\end{frame}

\begin{frame}{}
We have 
\begin{eqnarray}
size(x) &\ge& s_k \\
 & = & 2 + \sum_{i=2}^{k}s_{degree[y_i]} \\
 & \ge & 2 + \sum^k_{i=2}s_{i-2}
\end{eqnarray}
We then show by induction on $k$ that $s_k\ge F_{k+2}$, $\forall k\ge 0$. 
\end{frame}

\begin{frame}{}
Bases, $k=0$, $k=1$, trival. \\
Inductive step, assume that $k\ge 2$ and that $s_i\ge F_{i+2}$. 
\begin{eqnarray}
s_k &\ge& 2 + \sum^k_{i=2}s_{i-2} \\
 &\ge& 2 + \sum^k_{i=2} F_i \\
 & =& 1 + \sum^k_{i=0}F_i \\
 & = & F_{k+2}.
\end{eqnarray}
We conclude that $size(x)\ge s_k\ge F_{k+2}\ge \phi^k$. 
\end{frame}

\begin{frame}{}
\vspace{0.5cm}
{\it Corollary} The maximum degree $D(n)$ of any node in an $n$-node Fibonacci heap is $O(\lg n)$. \\
{\it Proof} $x$ a node in $n$-node Fibonacci heap and $k=degree[x]$. Above lemma says $n\ge$ $size(x) \ge \phi^k$. Taking 
base-$\phi$ logarithms results $k\le \log_\phi n$, i.e., the maximum degree $D(n)$ of any node is $O(\lg n)$. 
\end{frame}

\end{document}