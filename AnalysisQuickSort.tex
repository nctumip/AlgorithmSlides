\documentclass{beamer}
%\usepackage{amsmath,amssymb}
\usepackage{epsfig}
\usepackage{pdfpages}

\begin{document}
\begin{frame}{}

\vspace{1in}
\centerline{\Large Analysis of Quick Sort}
\vspace{0.5cm}
\centerline{Yu-Tai Ching}
\centerline{Department of Computer Science}
\centerline{National Chiao Tung University} 
\end{frame}

\begin{frame}{Quicksort Pseudo Code}

\begin{tabbing}
WW\=WW\=WW\=WW\= \kill \\
{\sc Quicksort} ($A$,$p$,$r$) \\
1. \> if $p<r$ \\
2. \> \> $q$ = {\sc Partition} ($A$,$p$,$r$) \\
3. \> \> {\sc Quicksort} ($A$,$p$,$r$) \\
4. \>\> {\sc Quicksort} ($A$,$q+1$,$r$) \\
\ \\
{\sc Partition}($A$,$p$,$r$) \\
1. \> $x=A[r]$ \\
2. \> $i=p-1$ \\
3. \> for $j=p$ to $r-1$ \\
4. \> \> if $A[j] \le x$; \\
5. \>\>\> $i=i+1$; \\
6.\>\>\> exchange $A[i]$ with $A[j]$; \\
7. \> exchange $A[i+1]$ with $A[r]$ \\
8. return $i+1$
\end{tabbing}
\end{frame}

\begin{frame}{Randomized Quick Sort}

\begin{tabbing}
WW\=WW\=WW\=WW\=WW\= \kill \\
{\sc Rabdomized-Partition} ($A$,$p$,$r$) \\
1. \> $i$ = {\sc Random} ($p$,$r$) \\
2. \> exchange $A[r]$ with $A[i]$ \\
3. \> return {\sc Partition} ($A$,$p$,$r$) \\
\ \\
{\sc Randomized-Quicksort} ($A$,$p$,$r$) \\
1. \> if $p<r$ \\
2. \>\> $q$ = {\sc Randomized-Partition} ($A$,$p$,$r$) \\
3. \>\> {\sc Randomized-Quicksort} ($A$,$p$,$r$) \\
4. \>\> {\sc Randomized-Quicksort} ($A$,$q+1$,$r$)
\end{tabbing}
\end{frame}

\begin{frame}{}

\begin{itemize}
\item Run time for {\sc Quicksort} = time spent in the {\sc Partition}
\begin{itemize}
\item {\sc Partition}, select a pivot, compare pivot with all of the others. 
\item pivot is never included in any future recursive call. 
\end{itemize}
\item Thus at most $n$ calls to {\sc Partition}, 
\item each {\sc Partition} takes a constant time plus the time required for the for loop (lines 3-6),
\item we count the number of times line 4 executed, comparing an element with pivot. 
\end{itemize}
\end{frame}

\begin{frame}{Lemma 7.1}

{\bf Lemma 7.1} Let $X$ be the number of comparisons performed in line 4 of {\sc Partition} over the entired execution of {\sc Quicksort} on an $n$-element array. Then the running time of {\sc Quicksort} is $O(n+X)$. \\ \ \\
{\bf Proof} There are at most $n$ calls of {\sc Partition}, each call take constant time and time for line 4. Total time for line 4 is $X$. \ \\ \ \\
Our goal is to compute $X$. 
\end{frame}

\begin{frame}{}

\begin{itemize}
\item Rename the elements of the array $A$ as $z_1, z_2, \ldots, z_n$. 
\item Define $Z_{ij} = \{ z_i,\ldots, z_j\}$ the set of element between $z_i$ and $z_j$. 
\item When does the algorithm compares $z_i$ and $z_j$?
\item Define an indicator random variable $$X_{ij} = I \{ z_i \mbox{ is compared to } z_j\}.$$.
\item Since each pair is compared at most once, total number of comparisons performed by the algorithm is $$X=\sum_{i=1}^{n-1}\sum_{j=i+1}^nX_{ij}.$$
\end{itemize}
\end{frame}

\begin{frame}{}

Taking expectation of both sides,a and then using linearity of expectation
\begin{align*}
E[X] &= E \left[  \sum_{i=1}^{n-1}\sum_{j=i+1}^nX_{ij} \right]\\
 &= \sum_{i=1}^{n-1}\sum_{j=i+1}^nE[X_{ij}] \\
 &= \sum_{i=1}^{n-1}\sum_{j=i+1}^n Pr \{ z_i \mbox{ is compared with } z_j \}
\end{align*}

\end{frame}

\begin{frame}{}

\begin{itemize}
\item We assumed that each pivot selected randomly. 
\item Observe that 
\begin{itemize}
\item Once a pivot $x$ is choosen, $z_i<x<z_j$, $z_i$ and $z_j$ cannot be compared any more. 
\item if $x==z_i$, $z_i$ is compared with all of the others. 
\item if $x==z_j$, $z_j$ is compared with all of the others.  
\end{itemize}
\item Thus $z_i$ and $z_j$ are compared iff $z_i$ or $z_j$ are selected as pivot. 
\item Probability $z_i$ ($z_j$) is selected as pivot  $$\frac{1}{j-i+1}$$ 
\end{itemize}
\end{frame}

\begin{frame}{}

 We conclude
\begin{align*}
Pr\{ z_i &\mbox{ is compared with }z_j\} \\
 &= Pr\{z_i \mbox{ or } z_j \mbox{ is selected as pivot from } Z_{ij}\} \\
 &= Pr\{z_i \mbox{ is pivot from }Z_{ij} \} + Pr \{ z_j \mbox{ is pivot from } Z_{ij}\} \\
 &= \frac{1}{j-i+1} +\frac{1}{j-i+1} \\
 &= \frac{2}{j-i+1} 
\end{align*}
\end{frame}

\begin{frame}{}

We get 
\begin{align*}
E[X] &= \sum_{i=1}^{n-1}\sum_{j=i+1}^n\frac{2}{j-i+1} \\
  &= \sum_{i=1}^{n-1}\sum_{k=1}^{n-i}\frac{2}{k+1} \ \ (\mbox{ let } k=j-i) \\
  &< \sum_{i=1}^{n-1}\sum_{k=1}^n\frac{2}{k} \\
  &= \sum_{i=1}^{n-1}O(\lg n) \ \ (\mbox{ by equation A. 7}) \\
  &= O(n\log n)
\end{align*}
\end{frame}
\end{document}
