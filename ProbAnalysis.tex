\documentclass{beamer}
%\usepackage{amsmath,amssymb}
\usepackage{epsfig}

\begin{document}
\begin{frame}{}

\vspace{1in}
\centerline{\Large Probabilistic Analysis}
\vspace{0.5cm}
\centerline{Yu-Tai Ching}
\centerline{Department of Computer Science}
\centerline{National Chiao Tung University} 
\end{frame}

\begin{frame}{Hiring Problem}

\begin{itemize}
\item Hire new office assistant through employment agency.
\item Agency provides one candidate a day, interview, decide either to hire or not. 
\item Cost
\begin{itemize}
\item Pay the agency a small fee to interview an applicant,
\item To actually hire an applicant is more costly, fire the current office assistant, pay substantial hiring fee to the agency.
\end{itemize}
\item Committed to have the best possible at all time, i.e., a new one is better than current one, then hire the new one.
\end{itemize}
\end{frame}

\begin{frame}{}

\begin{itemize}
\item What the price will be? 
\item Assume that candidates number 1 through $n$. 
\item Interview has low cost $c_i$, hiring has higher cost $c_h$.
\item If $m$ people hired, total cost $O(c_i\cdot n+c_h\cdot m)$. $c_i\cdot n$ is fixed, we analyze $c_h\cdot m$. 
\item Worst case, hire everyone interviewed, candidates come in strictly increasing order of quality. $O(c_h\cdot n)$. 
\end{itemize}
\end{frame}

\begin{frame}{Probabilistic Analysis}

\begin{itemize}
\item Must use knowledge of the distribution of the input, averaging total cost over all possible input, average case analysis. 
\item For the hiring problem, assume that the applicatns come in a random order. 
\item Meaning \begin{itemize}
\item Assume there is a total order for the qualification, 
\item i.e., candidate can be numbered $1, 2, \ldots, n$ and $rank(i)$ the rank of application $i$,
\item $<rank(1),rank(2), \ldots, rank(n)>$ a permutation of $<1,2,\ldots,n>$. 
\item Applicants come in random order is equivalent to saying that this list of ranks is equally likely to be any one of $n!$ permutation of $1,2,\ldots,n$, 
\item or the ranks from a uniform random permutation.
\end{itemize}
\end{itemize}
\end{frame}

\begin{frame}{Indicator Random Variable}

\begin{itemize}
\item A convenient method for converting between probabilities and expectation. 
\item Given a sample space $S$ and an event $A$, indicator random variable $I\{A\}$ associated with event $A$ defined as 
\[
I\{A\} = \left \{ 
\begin{array}{ll} 
1 & \mbox {if $A$ occurs,} \\ 
0 & \mbox {if  $A$ does not occurs.}  
\end{array}
\right.
\]
\item Expected number of heads obtained when flipping a fair coin
\begin{itemize}
\item sample space $\{H,T\}$, $Pr\{H\}=Pr\{T\}=1/2$,
\item define an indicator random variable $X_H$ Associated with the coin coming up head (event $H$),
\[
X_H=I\{H\}=
\left \{ 
\begin{array}{ll} 
1 & \mbox {if $H$ occurs,} \\ 
0 & \mbox {if  $H$ does not occurs.}  
\end{array}
\right.
\]
\end{itemize}
\end{itemize}
\end{frame}

\begin{frame}{}

\begin{itemize}
\item Expected number of heads obtained in one flip = the expected value of the indicator variable $X_H$,
\begin{align*}
E[X_H] &= E[I\{H\}] \\
 &= 1\cdot Pr\{H\} + 0\cdot Pr\{T\} \\
 &= \frac{1}{2}.
\end{align*}
\item Expected value of an indicator random variable assocoated with an event $A$ = the probability that $A$ occurs. 
\end{itemize}
\end{frame}

\begin{frame}{}

{\bf Lemma 5.1} Given a sample space $S$ and an event $A$ in the sample space $S$, let $X_A=I\{A\}$. Then $E[X_A] = Pr\{A\}$\\
\ \\
{\bf Proof} By the definition of the indicator random variable 
\[
I\{A\} = \left \{ 
\begin{array}{ll} 
1 & \mbox {if $A$ occurs,} \\ 
0 & \mbox {if  $A$ does not occurs.}  
\end{array}
\right.
\]
and the definition of the expected value, we have 
\begin{align*}
E[X_A] &= E[\{I\{A\}] \\
 &= 1.Pr\{A\}+0\cdot Pr\{\overline{A}\} \\
 &= Pr\{A\}. 
\end{align*}
where $\overline{A}$ denotes $S-A$, the complement of $A$. 
\end{frame}

\begin{frame}{}
\begin{itemize}
\item Indicator random variable useful for analyzing situations in which we perform repeated random trial. 
\item Let $X_i$ be the indicator random variable associated with the event in which $i$th flip comes up head,
\item $X_i = I\{ \mbox{the $i$th flip results in the event $H$  } \}.$
\item Let $X$ be the random variable denoting the total number of heads in the $n$ coin flips, $$X=\sum_{i=1}^{n}X_i.$$ 
\item Expected number of heads $$E[X] = E\left [\sum_{i=1}^{n}X_i\right ]$$
\end{itemize}
\end{frame}

\begin{frame}{}

By the linearity of expectation, 
\begin{align*}
E[X] &= E\left [\sum_{i=1}^n X_i\right ] \\
 &= \sum_{i=1}^n E\left [X_i\right ] \\
 &= \sum_{i=1}^n\frac{1}{2} \\
 &=\frac{n}{2}.
\end{align*}
\end{frame}

\begin{frame}{Now the Hiring Problem}
\begin{itemize}
\item Average cost = the expected number of times that we hire a new assistant. 
\item Assume that the candidates arrive in random order, 
\item Let $X$ be the random variable whose value equals the number of times we hire a new assistant, then $$E[X] = \sum_{x=1}^n x\cdot Pr\{X=x\}.$$
\item Difficult to compute $E[X]$. 
\end{itemize}
\end{frame}

\begin{frame}{Use Indocator Random Variable}
\begin{itemize}
\item Define $n$ variables related to whether or not each particular candidate is hired. 
\item Let $X_i$ be the indicator random variable associated with event in which the $i$th candidate is hired. 
\begin{eqnarray*}
X_i &=& I\{\mbox{candidate $i$ is hired}\} \\
 &=& 
 \left \{ 
 \begin{array}{ll} 
1 & \mbox {if candidate $i$ is hired,} \\ 
0 & \mbox {if  candidate $i$ is not hired.}  
\end{array}
\right.
\end{eqnarray*}
\item and $X=X_1+X_2+\ldots+X_n.$
\end{itemize}
\end{frame}

\begin{frame}{}

\begin{itemize}
\item By Lemma 5.1, $E[X_i] = Pr\{ \mbox{Candidate $i$ is hired.} \}$
\item What is the probability of candidate $i$ is hired? 
\item $i$ is hired iff $i$ is better that each of candidates 1 through $i-1$. 
\item Since candidates come arrive in random order, first $i$ candidates have appear in random order,
\item and  any one of the first $i$ candidates is equally likely to be the best-qualify so far,
\item candidate $i$ has a probability of 1/$i$ of being better quality that candidates 1 through $i-1$. 
\item and thus a probability of $1/i$ of being hired. $$E[X_i] = \frac{1}{i}.$$
\end{itemize}
\end{frame}

\begin{frame}{}

\begin{align*}
E[X] &= E\left [\sum_{i=1}^nX_i\right ] \\
 &=\sum_{i=1}^nE\left [X_i\right ] \\
 &=\sum_{i=1}^n\frac{1}{i} \\
 &= \ln n +O(1)
\end{align*}
Thus the average cost for hiring office assistant is $O(c_h\cdot \ln n)$
\end{frame}

\end{document}
