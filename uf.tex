\documentclass{beamer}
\usepackage{amsmath,amssymb}
\usepackage{epsfig}
\begin{document}

\begin{frame}{}
There are 7 cities, $b$, $c$, $d$, $e$, $f$, $g$, and $h$. \\
If there is a path connecting two cities, the pair of cities forms a pair. \\
$(b,e)$, $(h,c)$, $(f,d)$, $(c,e)$, $(g,f)$, $(e,h)$, and $(g,d)$ \\
Is there a path connecting any pair of cities? 
\end{frame}

\begin{frame}{}

\centerline{\Large Union-Find Operation}
\begin{itemize}
\item Group $n$ elements into a collection of disjoint sets. 
\item Two operations
\begin{itemize}
\item {\small Find which set a given element belongs to,}
\item {\small Union two sets.}
\end{itemize}
\item Find connected components, ({\small relation,
 reflexsive, symmetric, transtive}).
\end{itemize}
\end{frame}

\begin{frame}{}

\centerline{Disjoint-Set data structure}
\begin{itemize}
\item Maintain a collection $S=\{S_1,S_2,\ldots,S_k\}$ of disjoint dynamic
 sets.
\item Each Set is identified by representative- a member in the set.  
\end{itemize}
%\end{frame}

%\begin{frame}{}
\centerline{Operations}
\begin{itemize}
\item Make-Set($x$), create a new set, $x$ is the only member, $x$ is the
 representative.  
\item Union($x,y$), Unite the dynamic sets that contains $x$ and $y$ ($S_x$,
$S_y$). Representative of the union of $x$ and $y$ is some member of 
 $S_x\cup S_y$, $S_x$ and $S_y$ are destroied (removed from $S$). 
\item Find-Set($x$), return a pointer to the representative of the set
 containing $x$.   
\end{itemize}
\end{frame}

%\begin{slide}{}
%\centerline{Finding the Connected Components}
%\begin{itemize}
%\item Start with makeing set for each one, using Make-Set().
%\item If $i$ and $j$ are connected (related), Find-Set($i$) and Find-Set($j$),
%\item If they are in the same set, done,
%\item Otherwise, Union these two sets to be one set. Because $i$ and $j$
% make them connected. 
%\end{itemize}
%\end{slide}

\begin{frame}{}

\begin{itemize}
\item An application, determine the connected components of an 
 undirected graph.
\item Given $G=(V,E)$, 
\item for each $v\in V[G]$, do Make-Set($v$);
\item for each edge $(u,v)\in E[G]$, if Find-Set($u$) $\ne$ Find-Set($v$),
 then Union($u,v$).
\end{itemize}
%\end{frame}
\vspace{1cm}
%\begin{frame}{}
This is a problem, $n$ Make-Set operations, a sequence of $m$ Find-Set operations, 
 and there are at most $n-1$ Union operations.  \\
What is the data structure and what are the time complexcities for the
 operations.  
\end{frame}

\begin{frame}{}

\centerline{\large Linked-List representation}
\begin{itemize}
\item Head of the list is the representative.  
\item Every one has a pointer pointing to the representative. 
\item Make-Set, $\Theta(1)$ time. 
\item Find, $\Theta(1)$.
\item Union is little bit hard.  
\end{itemize}
\end{frame}

\begin{frame}{}

\epsfxsize 3.5in{
\centerline{\epsfbox{uf-ls.eps}}
}
\end{frame}

\begin{frame}{}
\centerline{\large Union in Linked-List Representation}
\begin{itemize}
\item Append one to the end of the other.
\item Change pointers to the head of the merged list.
\item There is a sequence of Unions that cost $\Theta(n^2)$ time if 
 there are $n-1$ union.  
\end{itemize}
\end{frame}

\begin{frame}{}

\centerline{\large A Weighted-Union Heuristic}
Append the smaller list onto the longer, tie broken arbitrary. \\
{\bf Theorem:} Using the linked-list representation of disjoint sets and the
 weighted-union heuristic, a sequence of $m$ Make-Set, Union, and 
Find-Set operations, $n$ of Make-Set operations, takes $O(m+n\lg n)$ time.
\end{frame}

\begin{frame}{}

Proof: 
Count the number of times that an object's representative
 pointer was modified.  \\
Consider an object $x$, first time its pointer was modified, the resulting
 set has at least 2 members.  \\
2nd time update, resulting set has at least 4 members.  \\
Updating $\lceil \lg k\rceil$ times, the resulting set has $k$ members.  \\
There are at most $\lceil \lg n\rceil$ updates for an object.  There are
$n$ objects, so Union costs at most $O(n\log n)$ time.  \\
For Make-Set and Find-Set, each one takes $O(1)$ and there are $m$
such operations.  Total cost is $O(m+n\lg n)$.
\end{frame}

\begin{frame}{}

\centerline{\Large Disjoint-set forest}
\begin{itemize}
\item Each member points only to its parent.
\item A connected component is a rooted tree.  
\item Parent pointer of the root points to itself.  
\item Root is the representative.  
\item Union, change Parent of the root of one set, Find-Set, chacing the
 Parent link.  
\end{itemize}
\end{frame}

\begin{frame}{}
\epsfxsize 3.5in{
\centerline{\epsfbox{uf-forest.eps}}
}
\end{frame}

\begin{frame}{}

\centerline{\large Time Complexity}
\begin{itemize}
\item Make-Set, each one takes $O(1)$ time. 
\item Union, each one takes $O(1)$ time.  
\item Find-Set, time required depending on the distance from the object
 to the root.  
\item There exists a sequence of $n$ Union/Find-Set operatine
 that takes $O(n^2)$ time.  
\end{itemize}
\end{frame}

\begin{frame}{}

\centerline{\large Heuristics to improve the running time}
\begin{itemize}
\item {\bf Union by rank}: Similar to the weighted-union heuristic, make the
 root of the tree with fewer nodes point to the root of the tree with more
 nodes. To ease the analysis, we maintain a {\bf rank} that is an
 upper bound on the height of the node.  Union by rank, the root with small
 rank is made to point to the root with larger rank during a {\sc Union}
 operation. 
\item {\bf Path Compression}: During the Find-Set operations, make each node
 on the find path point directly to the root.   
\end{itemize}
\end{frame}

\begin{frame}{}

\centerline{\large Pseudocode for disjoint-set forest}
{\sc Make-Set($x$)} \\
p[$x$] $\leftarrow x$ \\
$rank[x]$ $\leftarrow$ 0\\

\begin{tabbing}
WW\=WW\= \kill \\
{\sc Union $(x,y)$} \\
{\sc Link(Find-Set($x$),Find-Set($y$))} \\ 
\end{tabbing}

\begin{tabbing}
WW\=WW\= \kill \\
{\sc Find-Set($x$)} \\
{\bf if} $x\ne p[x]$ \\
\> {\bf then} $p[x]$ $\leftarrow$ {\sc Find-Set($p[x]$)}\\
{\bf return} $p[x]$
\end{tabbing}
\end{frame}

\begin{frame}{}

\begin{tabbing}
WW\=WW\=WW\=WW\=\kill \\
Link$(x,y)$ \\
{\bf if} $rank[x]>rank[y]$ \\
\>{\bf then} $p[y]\leftarrow x$ \\
\>{\bf else} $p[x]\leftarrow y$ \\
\>\>{\bf if} $rank[x]=rank[y]$ \\
\>\>\>{\bf then} $rank[y]\leftarrow rank[y]+1$\\
\end{tabbing}
\end{frame}

\begin{frame}{}

\centerline{\large Effect of the Heuristics on running time}
\begin{itemize}
\item If there are $f$ {\sc Find-Set} operations, the path-compression
 heuristic alone gives a worst case running time
$\Theta(f\log_{(1+f/n)}n)$ if $f\ge n$ and $\Theta(n+f\lg n)$ if $f<n$.
\item If both union by rank and path compression are applied, the wrost
case running time is $O(m\alpha(n))$ where $\alpha(n)$ is a {\it very}
 slowly growing function.
\item In any conceivable application of a disjoint-set data structure,
 $\alpha(n)\le 4$, we can view the running time as linear in $m$ in all
 practical situation. 
\end{itemize}
\end{frame}

\begin{frame}{}

\centerline{\large Functional Iteration}
{\small Section 3.2} \\
Notation $f^{(i)}(n):$ function $f(n)$ iteratively applied $i$
times to an intitial value on $n$. \\
Let $f(n)$ be a function over reals, for nonnegative integer $i$,
\begin{equation}
f^{(i)}(n) =\left \{
\begin{array}{ll}
n & {\rm \ if\ } i=0 \\
f(f^{(i-1)}(n)) & {\rm \ if\ }i>0
\end{array}
\right.
\end{equation}
If $f(n)=2n$, then $f^{(i)}(n)=2^in$. \\
For example $f^{(2)}(n)$ = $f(f^{(1)}(n))$ = $f(f(f^{(0)}(n)))$ = $f(f(n))$ = $f(2n)$ 
 = $2^2n$. 
\end{frame}

\begin{frame}{}

\centerline{\large Iterated logrithm function}
\begin{itemize}
\item $\lg ^*n$ denotes iterated logrithm.
\item $\lg^{(i)}n$ defined as above with $f(n)=\lg n$. 
\end{itemize}
\begin{equation}
\lg^* n = \min \{ i\ge 0| \lg^{(i)}n \le 1\}
\end{equation}
\end{frame}

\begin{frame}{}

$$
\begin{array}{rrr}
n & \lg n & \lg ^*n \\
2 & \lg 2 = 1 & \lg^*2 = 1 \\
4 & \lg 4 = 2 & \lg^*4 = 2 \\
16 & \lg 16 = 4 & \lg ^* 16 = 3\\
65536 & \lg 65536 = 16 & \lg ^* 65536 = 4\\
2^{65536} & \lg 2^{65536} = 65536 & \lg^* 2^{65536} = 5
\end{array}
$$
$\lg ^*n$ is a very very slow growing function. 

Since the number of observable universe is estimated to be about $10^{80}$
atoms, and $10^{80}<< 2^{65536}$. We rarely encounter an input size $n$
s.t. $\lg^*n>5$. 
\end{frame}

\begin{frame}{}

\centerline{\large Define the function $A_k(j)$}
For $k\ge0$ and $j\ge 1$,
\begin{equation}
A_k(j) = \left \{
\begin{array}{ll}
j+1 & {\rm \ if \ } k=0 \\
A_{k-1}^{(j+1)}(j) & {\rm \ if \ } k\ge 1
\end{array}
\right.
\end{equation}
where $A_{k-1}^{(j+1)}(j)$ functional-iterated notation,
\begin{eqnarray}
A_{k-1}^{(0)}(j) &=& j {\rm \ and}\\
A_{k-1}^{(i)}(j) &=& A_{k-1}(A_{k-1}^{(i-1)}(j)) {\rm \ for\ } i\ge 1.
\end{eqnarray}
$k$ the level of function $A$. 
\end{frame}

\begin{frame}{}

{\bf Lemma 2}: For any integer $j\ge 1$, we have $A_1(j)=2j+1$. \\
{\bf Proof} Induction on $i$ to show that $A_0^{(i)}(j)=j+i$. \\
Base: $A_0^{(0)}(j)=j=j+0$. \\
Assume that $A_0^{(i-1)}(j)=j+(i-1)$. \\
\begin{eqnarray}
A_0^{(i)}(j) &=& A_0(A_0^{(i-1)}(j))\\
 & = & A_0(j+(i-1)) \\
 & = & (j+(i-1))+1 \\
 & = & j+i.
\end{eqnarray}
\end{frame}

\begin{frame}{}

Finally
\begin{eqnarray} 
A_1(j) &=& A_0^{(j+1)}(j) \\
 &=& j+(j+1) \\
 &=& 2j+1
\end{eqnarray}
\end{frame}

\begin{frame}{}

{\bf Lemma 3}: For any integer $j\ge 1$, we have $A_2(j)=2^{(j+1)}(j+1)-1$. \\
{\bf Proof} Induction on $i$ to show that $A_1^{(i)}(j)=2^i(j+1)-1$. \\
For the base case: $A_1^{(0)}(j)=j=2^0(j+1)-1$. \\
For the inductive step: Assume that $A_1^{(i-1)}(j)=2^{i-1}(j+1)-1$. \\
Then 
\begin{eqnarray*}
A_1^{(i)}(j) &=& A_1(A_1^{(i-1)}(j)) \\
  &=& A_1(2^{(i-1)}(j+1)-1) \\
  &=& 2(2^{i-1}(j+1)-1)+1 \\
  &=& 2^i(j+1)-2+1 \\
  &=& 2^i(j+1) -1
\end{eqnarray*}
Finally 
\begin{eqnarray*}
A_2(j) &=& A_1^{(j+1)}(j) \\
  &=& 2^{j+1}(j+1)-1
\end{eqnarray*}
\end{frame}

\begin{frame}{}

\centerline{How quickly $A_k(j)$ grows, Look at $A_k(1)$ for level 0, 1, 2, 3, and 4.}
\begin{small}
\begin{eqnarray*}
A_0^{(1)} &=& 1+1 = 2, 
A_1(1) = 2\cdot 1 + 1 = 3 \\
A_2(1) &=& 2^{1+1}\cdot (1+1) - 1 = 7 \\
A_3(1) &=& A_2^{(2)}(1) = A_2(A_2(1)) = A_2(7) \\
 &=& 2^8\cdot 8 -1 \\
A_4(1) &=& A_3^{(2)} (1) = A_3(A_3(1)) = A_3(2047) \\
 &=& A_2 ^{(2048)}(2047) \\
 &>>& A_(2047)=2^{2048}\cdot2048-1 > 2^{2048} \\
 &=& (2^4)^{512} 
 = 16^{512} 
 >> 10^{80}
\end{eqnarray*}
\end{small}
Much greater than the estimated number of atoms in the observable unserve. 
\end{frame}

\begin{frame}{}

\centerline{The Inverse of $A_k{(n)}$, $n\ge 0$}
\begin{equation}
\alpha(n)=\min\{k: A_k(1)\ge n\}
\end{equation}
Or $\alpha (n)$ is the lowest level $k$, s.t. $A_k(1)$ is at least $n$. 
$$
\alpha(n) = \left\{
\begin{array}{ll}
0 & 0\le n \le 2 \\
1 & n=3 \\
2 & 4\le n \le 7 \\
3 & 8\le n \le 2047 \\
4 & 2048 \le n \le A_4(1)
\end{array}
\right.
$$
So $\alpha(n)\le 4$ for all practical purpose. 
\end{frame}

\begin{frame}{}

To show $O(m\alpha(n))$ bounds the running time of the disjoint-Set operation. 

{\bf Lemma 4}: $\forall$ nodes $x$, we have rank[$x$]$\le$rank$[p[x]]$ with strict inequality
$x\ne p[x]$; and from then on, rank$[x]$ does not change. The value of rank$[p[x]]$ 
monotonically increase over time. 

{\bf Corollary 5}: As we follow the path from any node toward a root, the node ranks 
strictly increase. 
\end{frame}

\begin{frame}{}

{\bf Lemma 6}: Every node has rank at most $n-1$. \\
{\small (A weak bound, a tight bound will be $\lfloor \lg n\rfloor$.) }\\
{\bf Proof} For each node, rank starts 0 (initialization, make a node a set). \\
It increases only upon {\sc Link} operation. There are $n-1$ {\sc Union} so
 there are at most $n-1$ {\sc Link} operations. \\
Each {\sc Link} either leaves all ranks alone or increase some nodes' rank by 1. \\
Thus all ranks are at most $n-1$.  \\
\end{frame}

\begin{frame}{}

\centerline{\Large Prove the Time Bound}

\centerline{\large Amortized Analysis}

%\centerline{\large Look at {\sc Link} operations}
\end{frame}

\begin{frame}{}

{\bf Lemma 7}: Suppose we convert a sequence of $S'$ of $m'$ {\sc Make-Set}, {\sc Union}, and
{\sc Find-Set} operations into a sequence $S$ of $m$ {\sc Make-Set}, {\sc Link}, and
{\sc Find-Set} by turing each {\sc Union} into two {\sc Find-Set} followed by a {\sc Link}.
The if sequence $S$ runs in $O(m(\alpha n))$ time, sequence $S'$ runs in $O(m'\alpha(n))$
time. \\
{\bf Proof} {\sc Union} in sequence $S'$ is converted into 3 operations in $S$. We have
$m'\le m\le 3m'$, thus $m=O(m')$. An $O(m\alpha(m))$ time bound for $S$ implies $O(m'\alpha(m))$
time bound for $S'$.
\end{frame}

\begin{frame}{}

\centerline{\large Potential Function}
\begin{itemize}
\item $\phi_q(x)$: a potential assigned to each node $x$ in the disjoint-set forest after $q$
 operations. 
\item $\Phi_q=\sum_x\phi(x)$: the potential for the entire forest after $q$ operations. 
\item The forest is empty prior the first operation, $\Phi_0=0$. 
\item $\phi_q(x)$: if $x$ is a tree root after the $q$th operation or if $rank[x]=0$,
 then $\phi_q(x)=\alpha(n)\cdot rank[x]$. 
\item If $x$ is not a root and $rank[x]\ge 1, \ldots$
\end{itemize}
\end{frame}

\begin{frame}{}

\centerline{\large $x$ is not a root and $rank[x]\ge 1$}
\ \\
\centerline{Define 2 auxiliary functions on $x$,}
\begin{equation}level(x)=\max \{k:rank[p[x]]\ge A_k(rank[x])\}\end{equation}
\begin{equation}iter(x)=\max\{ i:rank[p[x]]\ge A_{level(x)}^{(i)}(rank[x])\}\end{equation}
\end{frame}

\begin{frame}{}

$$level(x)=\max \{k:rank[p[x]]\ge A_k(rank[x])\}$$
$level(x)$: Greatest level $k$ for which $A_k$, applied to $x$'s rank, is
 no greater than $x$'s parent's rank. \\
{\bf Claim 1} $0\le level(x)<\alpha(n)$\\
{\bf proof} 
\begin{eqnarray*}
rank[p[x]] &\ge& rank[x]+1 {\rm \ \ by\ Lemma\ 4}\\
  &=& A_0(rank[x]) {\rm \ \ by\ definition\ of\ }A_0(j)
\end{eqnarray*}
That implies $level(x)\ge 0$. 
\begin{eqnarray*}
A_{\alpha(n)}(rank[x]) &\ge& A_{\alpha(n)(1)} \\
 &\ge& n \\
 & > & rank[p[x]]
\end{eqnarray*}
That implies $level(x)<\alpha(n)$. 
\end{frame}

\begin{frame}{}

$$iter(x)=\max\{ i:rank[p[x]]\ge A_{level(x)}^{(i)}(rank[x])\}$$
$iter(x)$: The largest number of times we can iteratively apply $A_{level(x)}$,
 applied to $x$'s rank before we get a value greater than $x$'s
 parent's rank. \\
{\bf Claim 2}: $1\le iter(x)\le rank[x]$ \\
{\bf Proof}: 
\begin{eqnarray*}
rank[p[x]] &\ge& A_{level(x)}(rank[x]) {\rm\ \ by\ definition\ of\ }level(x)\\
 &=& A_{level(x)}^{(1)} (rank[x]) {\rm\ \ by\ definition\ of\ functional\ iteration}
\end{eqnarray*}
So $iter(x)\ge 1$. 
\begin{eqnarray*}
A_{level(x)}^{(rank[x]+1)}(rank[x]) &=& A_{level(x)+1}(rank[x]) {\rm\ \ by\ definition\ of\ }A_k(j)\\
 &>& rank[p[x]] {\rm\ \ by\ definition\ of\ } level(x)
\end{eqnarray*}
which implies $iter(x)\le rank[x]$. 
\end{frame}

\begin{frame}{}

\centerline {\large Potential of node $x$ after $q$ operations}
\begin{small}
\begin{equation}
\phi_q(x)=\left\{
\begin{array} {ll}
\alpha(n)\cdot rank[x] & {\rm if\ }x{\rm\ is\ a\ root\ or\ }rank[x]=0 \\
(\alpha(n)-level(x))\cdot rank(x) - iter(x) & {\rm if\ }x{\rm\ is\ not\ a\ root\ and\ }rank[x]\ge 1
\end{array}
\right.
\end{equation}
\end{small}
\end{frame}

\begin{frame}{}

{\bf Lemma 8} For every node $x$, and for all operation count $q$,
$$0\le \phi_q(x)\le\alpha(n)\cdot rank[x]$$
{\bf Proof} If $x$ is a root or $rank[x]=0$, $\phi_q(x)=\alpha(n)\cdot rank[x]$.

If $x$ is not a root and $rank[k]\ge 1$:

To obtain lower bound on $\phi_q(x)$, we maximize $level[x]$ and $iter(x)$.
Since $level(x)\le\alpha(n)-1$ (Claim 1) and $iter(x)\le rank[x]$ (Claim 2)
\begin{eqnarray*}
\phi_q(x) &\ge& (\alpha(n)-(\alpha(n)-1))\cdot rank[x]-rank[x] \\
 &=& rank[x]-rank[x] \\
 &=& 0. 
\end{eqnarray*}
To obtain an upper bound on $\phi_q(x)$, we minimize $level(x)$ and $iter(x)$. 
Since $level(x)\ge 0$ (Calim 1), and $iter(x)\ge 1$ (Claim 2), we have
\begin{eqnarray*}
\phi_q(x) &\le& (\alpha(n)-0)\cdot rank[x] -1 \\
 &=& \alpha(n)\cdot rank[x]-1 \\
 &<& \alpha(n)\cdot rank[x]
\end{eqnarray*}
\end{frame}

\begin{frame}{}

{\bf Lemma 9} Let $x$ be a node that is not a root, suppose that $q$th operation is either a {\sc Link}
 or {\sc Find-Set}. Then after the $q$th operation, $\phi_q(x)\le \phi_{q-1}(x)$. \\
Moreover, if $rank[x]\ge 1$ and either $level(x)$ or $iter(x)$ changes due to the $q$th operation,
then $\phi_q(x)\le \phi_{q-1}(x)-1$. That is $x$'s potential cannot increase, and if it has positive
rank and either $level(x)$ or $iter(x)$ changes, then $x$'s potential drops by at least 1. 

{\bf Proof} Since $x$ is not a root, $q$th operation does not change $rank[x]$, and $n$ does not change after
 the initial $n$ {\sc Make-Set} operations, $\alpha(n)$ does not change wither. \\
If $rank[x]=0$, then $\phi_q(x)=\phi_{q-1}(x)=0$. \\
If $rank[x]\ge 1$: $level(x)$ monotonically increase \\
If $q$th operation does not change $level(x)$, $iter(x)$ either increases or remains unchanged. \\
If both $level(x)$ and $iter(x)$ are unchanged, then $\phi_q(x)=\phi_{q-1}(x)$. 
If $level(x)$ is unchanged and $iter(x)$ increase, $iter(x)$ increases by at least 1,
 $\phi_q(x)\le \phi_{q-1}(x)-1$. \\
If $q$th operation increases $level(x)$, it increases by at least 1. Thus $(\alpha(n)-level(x))\cdot rank[x]$
drops by at least $rank[x]$. \\
Since $level[x]$ increased, the value of $iter(x)$ might drop. The drop is at most $rank[x]-1$. (by Claim 2)\\
Thus, the increase in potential due to the change in $iter(x)$ is less than the decrease in potential
 due to the change in $level(x)$. \\
In all of the cases, $\phi_q(x)\le\phi_{q-1}(x)-1$. 
\end{frame}

\begin{frame}{}

{\bf Lemma 10} The amortized cost of each {\sc Make-Set} operation is $O(1)$. 

{\bf Proof} Suppose the $q$th operation is {\sc Make-Set}(x). This operation creates node
$x$ with rank 0, so that $\phi_q(x)=0$.
No other ranks or potentials change, so $\Phi_q=\Phi_{q-1}$. \\
The actual cost is of the {\sc Make-Set} operations is $O(1)$. 
\end{frame}

\begin{frame}{}

{\bf Lemma 11} The amortized cost of each {\sc Link} operation is $O(\alpha(n))$. 

{\bf Proof} The $q$th operation is {\sc Link}. The actually cost is the {\sc Link}
operation is $O(1)$. Wlog that the {\sc Link} makes $y$ the parent of $x$. \\
The nodes whose potentials may change are $x$, $y$, and the children of $y$. \\
To show that the only node whose potential can increase due to the {\sc Link}
is $y$, and the increase is at most $\alpha(n)$. \\
\begin{itemize}
\item By Lemma 9, any node of $y$'s children cannot have its potential increase due to the {\sc Link.}
\item $x$ was a root before $q$th operation, $\phi_{q-1}(x)=\alpha(n)\cdot rank[x]$. If $rank[x]=0$,
then $\phi_q(x)=\phi_{q-1}(x)=0$. Otherwise $\phi_q(x)=(\alpha(n)-level(x))\cdot rank[x]-iter(x)$
$<\alpha(n)\cdot rank[x]$.
\item $y$ was the root, $\phi_{q-1}(y)=\alpha(n)\cdot rank[y]$. $y$ is still a root after the {\sc Link}
operation. After the {\sc Link} operation, $y$'s rank increases by 1 or remains the same. Thus
either $\phi_q(y)=\phi_{q-1}(y)$ or $\phi_q=\phi_{q-1}(y)+\alpha(n)$. 
\end{itemize}
The amortized cost is $O(1)$ + $O(\alpha(n))$ = $O(\alpha(n))$. 
\end{frame}

\begin{frame}{}

{\bf Lemma 12} The amortized cost of each {\sc Find-Set} operation is $O(\alpha(n))$. 

{\bf Proof} $q$th operation is {\sc Find-Set} and that the find path contains $s$ nodes. The
actual cost is $O(s)$. \\
We show that \\
1. no node's potential increases due to the Find-Set and \\
2. at least $\max(0,s-(\alpha(n)+2))$ nodes on the find path have their potential decrease by at
 least 1. \\
``No node's potential increase" $\forall$ nodes except the root, by Lemma 9. \\
If $x$ is the root, then its potential is $\alpha(n)\cdot rank[x]$, which does not change. \\

To show that at least $\max(0,s-(\alpha(n)+2))$ node have their potential decrease by at least one. \\
$x$: a node on the find path that $rank[x]>0$ and $x$ is followed somewhere on the find path by another
 node $y$ that is not a root, where $level(y)=level(x)$ just before the {\sc Find-Set} operation.  \\
At most $\alpha(n)+2$ nodes on the path do not satisfy the constraints on $x$. The first node has
 rank 0, the last node is the root, and the last node $w$ on the path for which $level(w)=k$, $k=0, 1,
 2, \ldots, \alpha(n)-1$. \\

Fix such a node $x$. Let $k=level(x)=level(y)$. Before path compression, we have
\begin{eqnarray*}
rank[p[x]] &\ge& A_k^{(iter(x))}(rank[x]) ({\rm \ by\ definition\ of\ }iter(x)) \\
rank[p[y]] &\ge& A_k(rank[y]) ({\rm\ by\ definition \ of\ }level(y)) \\
rank[y] &\ge& rank[p[x]] {\rm\ (by\ Corollary\ 5)}
\end{eqnarray*} 
Let $i$ be the value of the $iter(x)$ 
\begin{eqnarray*}
rank[p[y]] &\ge& A_k(rank[y]) \\
 &\ge& A_k(rank[p[x]]) \\
 &\ge& A_k(A_k^{(iter(x))}(rank[x])) \\
 &=& A_k^{(i+1)}(rank[x])
\end{eqnarray*}
After path compression, $p[x]=p[y]$, thus $rank[p[x]]=rank[p[y]]$. $rank[p[x]]\ge A_k^{i+1}(rank[x])$.
$inter[x]$ increases to at least $i+1$, or $level(x)$ increases by 1 (if $inter(x)$ increases to at least
 $rank[x]+1$). In either case, we have $\phi_q(x)\le \phi_{q-1}(x)-1$. 

The amortized cost = actual cost plus the change in potential. Actual cost is $O(s)$.
Potential change (decrease) by at lease $\max(0,s-(\alpha(n)+2))$ which is at most
$O(s)-(s-(\alpha(n)+2))$ = $O(\alpha(n))$
\end{frame}

\begin{frame}{}

{\bf Theorem 13} A sequence of $m$ {\sc Make-Set}, {\sc Union}, and {\sc Find-Set} operations, $n$
of which are {\sc Make-Set} operations, can be performed on a disjoint-set forest with union by rank
 and path compression in worst-case time $O(m\alpha(n))$. 

{\bf Proof} By Lemmas 7, 10, 11, and 12. 
\end{frame}
\end{document}
\begin{frame}{}
\centerline{\Large Analysis }
\centerline{\Large of }
\centerline{\Large union by rank with path compression}
\centerline{\large Ackermann's function and its inverse}
\end{frame}

\begin{frame}{}
\centerline{Repeated exponentiation}
%\begin{itemize}
%\item 
For an integer $i\ge 0$, 
%$$ 2^{2^{\cdot^{\cdot^{\cdot^{2}}}}} $$
$$ 2^ {
\left.
2^{\cdot^{\cdot^{\cdot^{2}}}} 
\right \}i}$$
stands for the function $g(i)$, defined recursively by
$$
g(i) = 
\left \{
\begin{array}{lll}
2^1 & {\rm if } & i=0, \\
2^2 & {\rm if } & i=1, \\
2^{g(i-1)} & {\rm if } & i>1.
\end{array}
\right.
$$
The parameter $i$ gives the ``height of the stack of 2's'' that
make up the exponent.  
%\end{itemize}
\end{frame}

\begin{frame}{}
$$ 2^ {
\left.
2^{\cdot^{\cdot^{\cdot^{2}}}} 
\right \}4} = 2^{2^{2^{2^{2}}}} = 2^{65536}
$$
$g(i)$ is a function that grows very fast.
\end{frame}

\begin{frame}{}
\centerline{\large Define $\lg^*$ in terms of $\lg^{(i)}$, $i\ge 0$}
$$
\lg^{(i)} n = \left \{
\begin{array}{lll}
n & {\rm if \ }i=0, & \\
\lg(\lg^{(i-1)} n) & {\rm if \ }i>0 {\rm \ and\ }\lg^{(i-1)}n>0 \\
{\rm undefined\ }&{\rm if\ }i>0{\rm \ and\ }\lg^{(i-1)}n\le 0 {\rm \ or
 \ }\lg^{(i-1)}n{\rm \ is\ undefined};
\end{array}
\right.
$$
$ \lg^*n=\min\{i\ge 0:\lg^{(i)}n\le 1\} $
$\lg^* n$ function is the inverse of repeated exponentation
$$
 2^{\left. 2^{\cdot^{\cdot^{\cdot^{2}}}}\right \}k} =n, \ \ \ 
\lg^* 
 2^{\left. 2^{\cdot^{\cdot^{\cdot^{2}}}}\right \}k} = k+1
$$
\end{frame}

\begin{frame}{}
\centerline{\bf Ackermann's function}
For integers $i,j\ge 1$
$$
\left .
\begin{array}{llll}
A(1,j) & = & 2^j & {\rm for\ }j\ge 1, \\
A(i,1) & = & A(i-1,2) & {\rm for\ }i\ge 2, \\
A(i,j) & = & A(i-1,A(i,j-1)) & {\rm for\ }i,j\ge 2.
\end{array}
\right .
$$
\end{frame}

\begin{frame}{}
\begin{tabular}{l | llll}
      &  $j=1 $  & $j=2$ & $j=3$ & $j=4$ \\ \hline
$i=1$ &  $2^1$   & $2^2$ & $2^3$ & $2^4$ \\
$i=2$ & $2^2$ & $2^{2^2}$ & $2^{2^{2^2}}$ & $2^{2^{2^{2^2}}}$ \\
$i=3$ & $2^{2^2}$ & $2^{\left. {2^{\cdot^{\cdot^{\cdot^2}}}}\right \} 16}$ & & 
\end{tabular}
\begin{itemize}
\item $A(i,j)$ grows much faster than $2^{{\cdot^{\cdot}}\}j}$ \\
\item The inverse of the Ackermann's function 
$$ \alpha(m,n) = \min\{i\ge 1: A(i,\lfloor m/n\rfloor) > \lg n\}$$ 
\end{itemize}
\end{frame}

\begin{frame}{}
\centerline{\bf Observation}
\begin{itemize}
\item If we fix $n$, then as $m$ increases, the function $\alpha(m,n)$ is
 monotonically decrease. 
  \begin{itemize}
  \item $\lfloor m/n\rfloor$ monotocnically increasing as $m$ increasing.  
  \item since $n$ is fixed, the smallest $i$ needed to bring
    $A(i,\lfloor m/n\rfloor)$ above $\lg n$ monotonically decreasing as $m$
    increasing, 
    (thuse $\alpha(m,n)$ monotonically decreasing).  
  \end{itemize}
\item Given $n$ elements, as the number of operations $m$ increasing,
 we would expect the average find-path length to decrease due to
 path compression.  
\end{itemize}
\end{frame}

\begin{frame}{}
\centerline{\large Why $\alpha(m,n)\le 4$ for all practical purpose?}
\begin{itemize}
\item $A(i,\lfloor m/n\rfloor) \ge A(i,1)$ since $m/n\ge 1$.
\item $A(4,\lfloor m/n\rfloor) \ge A(4,1)$.
\item $A(4,1)=A(3,2)$ = $2^{\left.{2^{\cdot^{\cdot^{\cdot^{2}}}}}\right\}16 }$.
\end{itemize}
\end{frame}

\begin{frame}{}
\centerline {\large \bf To show the $O(m\lg^* n)$ bound}
{\it \bf Lemma 22.2} For all nodes $x$, we have $rank[x]\le rank[p[x]]$,
with strictly inequality if $x\ne p[x]$.  The value of $rank[x]$ is
 initially 0 and incrases through time until $x\ne p[x]$; from then on,
 $rank[x]$ does not change.  The value of $rank[p[x]]$ is a monotonically
 increasing function of time.  
\end{frame}

\begin{frame}{}
{\it \bf Lemma 22.3} For all tree roots $x$, size($x$)$\ge 2^{rank[x]}$. \\
{\bf Proof} The proof is by induction on the number of {\sc Link} 
 operations.  \\
Basis: The lemma is true before the first {\sc Link}, since ranks are 
 initially 0 and each tree contains at least one node.  \\
Inductive step: Assume that the lemma holds before performing the operation
 {\sc Link}($x,y$).  Let $rank$ ($size$)
 denote the the rank (size) just before the {\sc Link},
 and let $rank'$ ($size'$) denote the rank (size) just after the {\sc Link}.\\
If $rank[x]\ne rank[y]$, assume wlog that $rank[x]<rank[y]$.  $y$ is the
 root of the tree formed by the {\sc Link} operation, and \\
\end{frame}

\begin{frame}{}
\begin{eqnarray*}
size'(y) &=& size(x)+size(y) \\
 &\ge& 2^{rank[x]}+2^{rank[y]} \\
 &\ge& 2^{rank[y]} \\
 &=& 2^{rank'[y]}.
\end{eqnarray*}
Note: no ranks or sizes change for any nodes other than $y$. \\
If $rank[x]=rank[y]$, node $y$ is root of the new tree,
\begin{eqnarray*}
size'(y) &=& size(x)+size(y) \\
 &\ge& 2^{rank[x]}+2^{rank[y]} \\
 &=& 2^{rank[y]+1} \\
 &=& 2^{rank'[y]}
\end{eqnarray*}
\end{frame}

\begin{frame}{}
{\it \bf Lemma 22.4} For any integer $r\ge 0$, there are at most
 $\frac{n}{2^r}$ nodes of rank $r$. \\
{\it \bf Proof}
Fix a particular value of $r$.  
When we assign a rank $r$ to a node $x$, we attach a label $x$ to
 all nodes in the tree rooted $x$.  \\
There are at least $2^r$ are labeled each time.  \\
If the root of the tree containing node $x$ changes, rank of the 
new root is at least $r+1$.  Since we assign labels only when 
a root is assigned a rank $r$, no nodes in the new tree is labeled.  \\
Thus each node is labeled at most once, when its root is first assigned
 rank $r$.  
There are at most $n$ labeled nodes.  There are at least $2^r$ labels
 assigned for each node of rank $r$. \\
If there were more than $n/2^r$ nodes of rank $r$, we would have
 more than $2^r\cdot\frac{n}{2^r}$ nodes are labeled, a contradiction.  
\end{frame}

\begin{frame}{}
{\bf \it Corollary 22.5} Every node has rank at most $\lfloor \lg n\rfloor$.\\
{\it \bf Proof} If $r>\lg n$, then there are at most $n/2^r <1$ nodes
 of rank $r$.  Since rank are natural numbers, corollary follows.  
\end{frame}

\begin{frame}{}
{\bf \it Lemma 22.6} Suppose we convert a sequence $S'$ of $m'$ {\sc Make-Set},
{\sc Union}, and {\sc Find-Set} in to a sequence $S$ of $m$ {\sc Make-Set},
{\sc Union}, and {\sc Find-Set} by turing each {\sc Union} into two 
{\sc Find-Set} operations followed by a {\sc Link}.  Then if sequence
 $S$ runs in $O(m\lg^*n)$ time, sequence $S'$ runs in $O(m'\lg^*n)$ times. \\
{\it \bf Proof} $m'\le m\le 3m'$.
\end{frame}

\begin{frame}{}
{\it\bf Theorem 22.7} A sequence of $m$ Make-Set, Link, and Find-Set
 operations, $n$ of which are Make-Set, can be performed on a disjoint-set
forest with union by rank and path compression in worst-case time $O(m\lg^*n)$.
\\
{\it \bf Proof}
The charges assessed to Make-Set and Link are $O(n)$ since each one
takes $O(1)$ and there are $n$ Make-Set and $n-1$ Link.  \\
To analysis the cost for Find-Set operations \\
partition node ranks inot {\it\bf blocks} by putting rank $r$ into
block $\lg^*r$ for $r=0,1,\ldots,\lfloor\lg n\rfloor$.  \\
for convenience, define for integers $j\ge -1$,
\end{frame}

\begin{frame}{}
for convenience, define for integers $j\ge -1$,
$$
B(j)=\left \{
\begin{array}{lll}
-1 & {\rm if} & j=-1, \\
1  & {\rm if} & j=0, \\
2  & {\rm if} & j=1, \\
2^{\left. 2^{\cdot^{\cdot{\cdot{2}}}} \right\}j-1} &{\rm if} & j\ge 2. 
\end{array}
\right.
$$
for $j=0,1,\ldots,\lg^*n-1$, the $j$th block consists of the set of
ranks $\{B(j-1)+1,B(j-1)+2,\ldots,B(j)\}$.
\end{frame}

\begin{frame}{}
\begin{eqnarray*}
B(-1) &=& -1 \\
B(0) &=& 1 \\
B(1) &=& 2 \\
B(2) &=& 4 \\
B(3) &=& 16 \\
B(4) &=& 2^{16}=65536 \\
B(5) &=& 2^{65536} \\
\lg^*n & & n
\end{eqnarray*}
\end{frame}

\begin{frame}{}
Two types of the charges for a Find-Set operation: {\it\bf block charges}
and {\it\bf path charges}.   
\begin{small}
\begin{itemize}
\item block charge: A Find-Path finds a path $x_0,x_1,\ldots,x_l$.
For $j=0,1,\ldots,\lg^*n-1$, assess one block charge to the last node
 with rank in block $j$ on the path.  Also assess one block charge to the
child of the root, i.e., to $x_{l-1}$.  
\item Or we say assess one block charge to $x_i$ that $p[x_i]=x_l$ or 
 $\lg^*rank[x_i]<\lg^*rank[x_{i+1}]$.
\item path charge: each node on the find path for which we do not assess
 a block charge, we assess one path charge.  
\item Total cost is block charges + path charges.  
\end{itemize}
\end{small}
\end{frame}

\begin{frame}{}
{\bf Once a node other than the root ot its child is assessed block charge,
it will never again be assessed path charges.} 
\begin{small}
\begin{itemize}
\item When path compression, rank of $x_i$, $p[x_i]\ne x_l$, does not change.
\item rank of new parent of $x_i$ is strictly greater than old parent of $x_i$.
%(because $x_i$ is the last node in block $j$.) 
\item difference between $\lg^*rank[p[x_i]]$ and $\lg^*rank[x_i]$ is
 monotonically increasing function of time.  
\item once $x_i$ and its parent have rank of different vlocks, they will 
 always have ranks in different blocks, thus no block charge any more.  
\end{itemize}
\end{small}
\end{frame}

\begin{frame}{}
\centerline{\bf To bound the block charge}
\begin{itemize}
\item there is a charge for each block
\item there are 0 to $\lg^*n-1$ blocks thus at most $\lg^*n+1$ block charges 
 for each Find-Set.  
\item there are at most $m(\lg^*n+1)$ block charges.  
\end{itemize}
\end{frame}

\begin{frame}{}
\centerline{\bf To Bound the Path Charges}
\begin{itemize}
\item To maximize the path charge for a node $x_i$.  
\item After pathe compression $p[x_i]$ and $x_i$ are in the same block so
 $x_i$ could receive next path charge.  
\item to maximize, $x_i$ has lowest rank in its block, namely $B(j-1)+1$, 
 its parents' rank successively take on the value $B(j-1)+2$,  $B(j-1)+3$,
\dots, B(j).  
\item There are $B(j)-B(j-1)-1$ such ranks, then each vertex can has this much
 path charges.  
\end{itemize}
\end{frame}

\begin{frame}{}
{\bf Next, bound the number of nodes that have ranks in block $j$, $j>0$}  \\
$N(j)$, the number of nodes whose ranks are in block $j$.  By Lemma 22.4
$$
N(j) \le \sum^{B(j)}_{r=B(j-1)+1}\frac{n}{2^r}.
$$
$j=0$
\begin{eqnarray*}
N(0) &=& \frac{n}{2^0} + \frac{n}{2^1} \\
  &=& \frac{3n}{2} \\
  &=& \frac{3n}{2B(0)}.
\end{eqnarray*}
\end{frame}

\begin{frame}{}
For $j\ge 1$,
\begin{eqnarray*}
N(j) &\le& \frac{n}{2^{B(j-1)+1}}\sum_{r=0}^{B(j)-(B(j-1)+1)}\frac{1}{2^r} \\
  &<& \frac{n}{2^{B(j-1)+1}}\sum_{r=0}^\infty\frac{1}{2^r}\\
  &=& \frac{n}{2^{B(j-1)}} \\
  &=& \frac{n}{B(j)}.  
\end{eqnarray*}
We conclude $N(j)\le \frac{3n}{2B(j)}, \forall j\ge 0$.  
\end{frame}

\begin{frame}{}
{\bf To finish bounding the path charges}, \\
summing over all blocks the product of (maximum number of nodes with ranks
 in the block) and (the maximun number of path charges per node of that
 block). \\
$P(n)$, the overall number of path charges
\begin{eqnarray*}
P(n) &\le & \sum_{j=0}^{\lg^*n-1}\frac{3n}{2B(j)}(B(j)-B(j-1)-1) \\
  &\le &\sum_{j=0}^{\lg^*n-1}\frac{3n}{2B(j)}\cdot B(j) \\
  &=& \frac{3}{2}n\lg^*n.  
\end{eqnarray*}
\end{frame}

\begin{frame}{}
Total number of charges for Find-Set operations, $O(m(\lg^*n+1)+n\lg^*n)$.
which is $O(n\lg^*n)$ since $m>n$. \\
This concludes the proof.  
\end{frame}
\end{document}