\documentclass{beamer}
%\usepackage{amsmath,amssymb}
%\usepackage{epsfig}
\usepackage{pdfpages}
\begin{document}
\begin{frame}{}

\vspace{1in}
\centerline{\Large Red Black Tree}
\vspace{0.5cm}
\centerline{Yu-Tai Ching}
\centerline{Department of Computer Science}
\centerline{National Chiao Tung University} 
\end{frame}

\begin{frame}{Red Black}

\begin{itemize}
\item Each node is colored, either red or black. 
\item A kind of height balance tree. 
\item Inplementation, each node contains the attributes, {\it color}, {\it key}, {\it left}, {\it right}, and {\it p}. 
\item If a node does not have {\it p}, {\it left}, or {\it right}, they point to {\sc Nil}
\item All {\sc Nil}s are replaced by a black node {\sc T.Nil}. 
\end{itemize}
\end{frame}

\begin{frame}{Red Black Tree Definition}
\begin{enumerate}
\item Every node is either red or black. 
\item The root is black. 
\item Every leaf is black {\sc Nil}.
\item If a node is red, then both its children are black. 
\item For each node, all simple paths from the node to decendant leaves contains the same number of vlack nodes.  
\end{enumerate}
\begin{itemize}
\item Figure 13.1.
\item The black height of node $x$, $bh(x)$: number of black nodes on simple paths from node $x$ (not including $x$) down to a leaf. 
\item Black height of a RB Tree, the black height of the root. 
\item Together with property 4, height of RB Tree is at most 2*$bh(root)$. 
\end{itemize}
\end{frame}

\begin{frame}{}

{\bf Lemma 13.1 } RB Tree with $n$ internal nodes has height at most $2\cdot \lg(n+1)$.
\begin{itemize}
\item Immediate consequence, {\sc Search}, {\sc Minimum}, {\sc Maximum}, {\sc Successor}, {\sc Predecessor} can be done in $O(\lg n)$ time,. 
\item How about {\sc Insertion} and {\sc Deletion}? Need to change color of some nodes and rotation. 
\item figures of rotations, left and right rotation. 
\end{itemize}
\end{frame}

\begin{frame}{BST Quick Review}

\begin{itemize}
\item Insertion, 
\item Deletion, $z$ is to be deleted, 
\begin{itemize}
\item $z$ has one child or less, delete it. 
\item $z$ has two children, move the {\sc Successor($z$)} to $z$'s place, delete {\sc Successor($z$)}. 
\end{itemize}
\end{itemize}
\end{frame}

\begin{frame}{RB Insertion}

\begin{itemize}
\item Insert a node $s$ as insertion $z$ into a binary search tree. 
\item $z$ is inserted as a red node. 
\item If $z.p$ is black, we are done. 
\item If $z.p$ is red, we have consecutive red nodes. 
\item Two cases. Either $z$ is left child or right child of $z.p$, cases depends on the color of $z$'s uncle, $y$.
\end{itemize}
\end{frame}

\begin{frame}{Case 1, $y$ is red}

\begin{itemize}
\item $z.p.p$ is black (why?). 
\item swap color of the two levels, $z.p.p$  was black, now it becomes red,
\item $z.p$ and its sibling $y$ were red, now they are black. 
\item Check RB Tree properties. 
\item $z.p.p$ is the new $z$, violation if $z.p$ is red. 
\end{itemize}
\end{frame}

\begin{frame}{Case 2, $y$ is black}

\begin{enumerate}
\item 2-1 $z$ is right child of $z.p$,
\begin{itemize}
\item left rotate about A (Figure 13.6) to get case 2-2
\end{itemize}
\item 2-2 $z$ is left child of $z.p$. 
\begin{itemize}
\item color changes and right rotate about C (Figure 13.6), we are done. 
\item check red-black property, and why we are done?
\end{itemize}

\end{enumerate}
\end{frame}

\begin{frame}{RB Tree Delete}

\begin{itemize}
\item $z$ is deleted, 
\item $z$ has one child or less, its child $x$ moves to $z$'s place ({\it left} or {\it right} of $z.p$ points to $z$'s child). 
\item We push the color of $z$ to $x$. 
\item If $z$ was red, $x$ must be black, we are done. Red is removed, $x$ now is black. 
\item  If both $z$ and $x$ were black, we have a double black $x$. 
\item $z$ has two children, $z$'s successor $w$ is moved to $z$'s place, and take $z$'s color.  $w$ is deleted, we have previous case. 
\item We have to take care the case that $x$ is double black. 
\item A simple case, if $x$ is the root, we simply remove the extra black. 
\end{itemize}
\end{frame}

\begin{frame}{$x$ is doubly black, $x$ is not root}

\begin{itemize}
\item Assume that $x$ is left child of $x.p$, the other case is symmetric. 
\item Two cases depend on the color of $w$, the sibling of $x$
\item $w$ is red, we have case 1. If $w$ is black, we have some more sub-cases. 
\end{itemize}
\end{frame}

\begin{frame}{Case 1, $w$ is red, Figure 13.7 (a)}

\begin{itemize}
\item $x.p$ must be black, 
\item children of $w$ must be black. 
\item change color of B and D, and left rotate about B,
\item $x$'s sibling is now $C$, $C$ is black, we have case 2.  
\end{itemize}
\end{frame}

%\begin{frame}
%\includepdf[pages={1}]{Figure137a.pdf}
%\begin{figure} \centering {
%\includegraphics{Figure137a.pdf}
%}
%\end{frame}

\begin{frame}{Case 2, $w$ is black, both $w$'s children are black, Figure 13.7 (b)}

\begin{itemize}
\item $x.p$ can be red or black,  children of $w$ are black as defined. 
\item push one black from $x$ and one black from $w$ to $x.p$, $x$ becomes single black, $w$ becomes red,
\item color of $x.p$ depends on the original color, 
\begin{itemize}
\item $x.p$ was red, then $x.p$ becomes black. 
\item $x.p$ was black, the $x.p$ becomes double black, it is now the new $x$. 
\end{itemize}
\end{itemize}
\end{frame}

\begin{frame}{Case 3, $w$ is black, one of $w$'s children is black and the other is red}
Sub-case 3.1, left child is red. {\color{blue} Figure 13.7 (c)} 
\begin{itemize}
\item Right rotate about $D$, and change color,
\item now $C$ is new $w$, right child is red, we have sub-case 3.2,
\end{itemize}
\end{frame}

\begin{frame}{}

Sub-case 3.2, right child is red. {\color{blue} Figure 13.7 (d)}
\begin{itemize}
\item Color changes (refer to procedure {\sc RB-DEKETE-FIXUP},
\item  left rotate about $B$, 
\end{itemize} 
\end{frame}

\end{document}
